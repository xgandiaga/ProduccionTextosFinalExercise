\documentclass[../Final.tex]{subfiles}
\begin{document}

Impact can be described as an applied high-magnitude load in short-time period. Involved objects that undergo this load susceptible to experience failure as most of instruments are 
designed to withstand gradual energy and load. This statement explicitly clarify the negative effect of impact phenomenon which closely related with failures and casualties. 
In naval architecture field, this load was taken seriously after catastrophe of Exxon Valdez which massively contributed to destruction of Alaska's water territory, after received impact load in form of grounding. 
In other case, remarkable human life losses happened as anticipation of pas­senger evacuation in tragedy of the Titanic could not be fulfilled in event of collision impact. 
Efforts to increase safety have been conducted in various ways, including in calculation method to predict casualties on involved objects. Concept of collision energy was introduced by Minorsky \cite{minorsky1958analysis} 
in 1959 who concerned regarding safety of nuclear reactor on ship. His for­mula was followed by analytical concept of collision by Zhang \cite{zhang1999mechanics}. 
Even though the concept was simplified, but it was considered useful to estimate ship positions during collision. Recent development of ship collision was expanded into collision between structure and iceberg by Liu and Amdahl \cite{liu2010new}. 
The concept of contact between two entities by Stronge \cite{coaplen2004work} was developed analytically and the results was compared with finite element simulation. 

As other physical phenomenon, numerous influences which come from internal or external parts can affect the result of impact between two entities or even with more subjects. 
In this paper, influence of structural configuration and mechanical properties of material were taken into consideration as applied parameters which were included in the internal parts. 
Double hull size with different width represented structure configuration, while material properties took strength characteristic, strain behaviour, and hardening parameter as selected parameters. 
Structural responses, including energy, force, and damage were evaluated based on applied parameters. 
Contribution of applied parameter would be compared in order to obtain information regarding influence level to the observed responses on the struck ship. 

\end{document}