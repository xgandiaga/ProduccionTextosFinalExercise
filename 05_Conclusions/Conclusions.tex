\documentclass[../Final.tex]{subfiles}
\begin{document}
This work presented a study on structural and material aspects of double hull passenger ship as the struck ship against certain impact phenomenon. 
Collision with the striking ship was taken into consideration in this study with two ships involved in the proposed scenario. Discussion was initiated on structural aspect, which two different double hull sizes were compared. 
Results indicated that narrower the distance between the outer and inner hulls would deliver more significant effect to inner hull which could harm cargo safety. 
Implementation of wider double was considered good option to increase safety, but capacity of the ship would be reduced as consequences. 
Double hull with space 1.5 m was used in further collision analysis with different material configurations were applied. Strength characteristic was concluded deliver the most significant effect to the calculation results. 
Response of energy and force during collision supported this argument. Implementation of high strength material can be implemented on the side hull to give better resistance against collision load. 
However, treatment of this material should be given serious attention as remarkable treatment cost may occur. Failure strain in other hand addressed gradual effect to internal energy as higher stain constant was used in analyses. 
Larger strain constant made the struck ship experienced wrinkle on the area near contact surface. The structure was also found harder to be torn by the striking ship. 
Assumption in yield surface was represented by hardening parameter, which was considered contribute the lowest effect to the calculation results. 
This study can be reasonable reference in applying material behaviour for impact analysis, especially for marine structures under collision and grounding. 
Further considerations in conducting material testing and laboratory experiment for nonlinear and large scale analyses can be performed to increase accuracy of material property and behaviour as they are highly encouraged by the authors. 
\end{document}